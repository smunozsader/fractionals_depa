% === SECCIÓN 11: DISPOSICIONES FINALES ===

\Clause{Disposiciones Finales}

\SubClause{Artículo 24. Transferencia de Derechos Fiduciarios}

\textbf{24.1 Derecho de preferencia}

Si un FIDEICOMISARIO desea vender su derecho fiduciario (25\% de la propiedad), debe:

\begin{enumerate}[label=\arabic*.]
  \item \textbf{Notificar por escrito} a los otros tres (3) FIDEICOMISARIOS especificando:
  \begin{itemize}
    \item Precio de venta propuesto.
    \item Términos y condiciones.
    \item Tiempo máximo para manifestar interés (mínimo 30 días).
  \end{itemize}
  
  \item \textbf{Derecho de preferencia:} Los demás FIDEICOMISARIOS tienen opción exclusiva de compra al mismo precio y términos dentro de plazo establecido.
  
  \item Si ninguno desea comprar, el FIDEICOMISARIO puede vender a tercero, sujeto a:
  \begin{itemize}
    \item Aprobación del FIDUCIARIO (no puede ser irrazonablemente negada).
    \item El nuevo FIDEICOMISARIO debe aceptar este contrato y reglamento.
    \item Nuevo FIDEICOMISARIO debe aportar \$900,000 MXN (neto de comisiones si aplica).
  \end{itemize}
  
  \item \textbf{Transferencia registral:} Se formaliza mediante cambio de titularidad ante notario e inscripción en Registro Público.
\end{enumerate}

\textbf{24.2 Herencia}

El derecho fiduciario es transmisible a los herederos legales del FIDEICOMISARIO fallecido. En este caso:
\begin{enumerate}[label=\arabic*.]
  \item Los herederos adquieren automáticamente los derechos y obligaciones.
  \item Deben ratificar este contrato ante notario dentro de 90 días.
  \item Si rechazan la herencia, aplica derecho de preferencia para demás FIDEICOMISARIOS.
\end{enumerate}

\SubClause{Artículo 25. Causales de Resolución del Fideicomiso}

El FIDEICOMISO se resuelve por:

\begin{enumerate}[label=\arabic*.]
  \item \textbf{Vencimiento del plazo:} 50 años (ver Artículo 4).
  \item \textbf{Acuerdo unánime:} Los 4 FIDEICOMISARIOS pueden decidir resolver anticipadamente.
  \item \textbf{Pérdida total:} Si el inmueble se destruye totalmente y no es recuperable.
  \item \textbf{Incumplimiento grave:} Si el FIDUCIARIO incumple sustancialmente sus obligaciones y no subsana dentro de 60 días de notificación.
\end{enumerate}

Al momento de resolución:
\begin{itemize}
  \item Los derechos fiduciarios se consolidan en dominio directo de FIDEICOMISARIOS.
  \item Se vende la propiedad o se regresa al FIDEICOMITENTE (según aplicable).
  \item Se liquidan pasivos y se distribuyen sobrantes.
\end{itemize}

\SubClause{Artículo 26. Normatividad Aplicable}

Este FIDEICOMISO se rige por:

\begin{enumerate}[label=\arabic*.]
  \item \textbf{Ley General de Títulos y Operaciones de Crédito} (artículos 381-410).
  \item \textbf{Ley de Inversión Extranjera} (si aplica a fideicomisarios extranjeros).
  \item \textbf{Código Civil Federal} y \textbf{Código Civil del Estado de Yucatán}.
  \item \textbf{NOM-029-SE-2021:} ``Prácticas comerciales—Requisitos informativos para la prestación del servicio de tiempo compartido''.
  \item \textbf{Leyes tributarias:} ISR, IVA, impuestos locales aplicables.
  \item \textbf{Ordenanzas municipales de Mérida, Yucatán.}
  \item \textbf{Contrato de este Fideicomiso} y \textbf{Reglamento Interno}.
\end{enumerate}

\SubClause{Artículo 27. Jurisdicción y Competencia}

Las partes se someten a la jurisdicción de los Juzgados Federales en Materia Civil del Distrito Judicial de Mérida, Yucatán, renunciando a cualquier otro fuero que pudiera corresponderles.

En caso de controversia se agotarán previamente:
\begin{enumerate}[label=\arabic*.]
  \item \textbf{Mediación:} Gestión mediante mediador neutral.
  \item \textbf{Arbitraje:} Si mediación falla, las partes someten a arbitraje conforme a Reglas de Arbitraje de la Cámara de Comercio de Yucatán.
  \item \textbf{Litigio judicial:} Si arbitraje es inviable.
\end{enumerate}

\SubClause{Artículo 28. Confidencialidad y Privacidad}

Las partes se comprometen a:
\begin{enumerate}[label=\arabic*.]
  \item Mantener confidencial información financiera del fideicomiso.
  \item No divulgar datos personales de otros FIDEICOMISARIOS.
  \item Respetar la privacidad de huéspedes en la propiedad.
  \item Cumplir con regulaciones de protección de datos personales.
\end{enumerate}

\SubClause{Artículo 29. Modificaciones al Contrato}

Este contrato puede modificarse únicamente:

\begin{enumerate}[label=\arabic*.]
  \item Con consentimiento escrito de \textbf{TODOS los FIDEICOMISARIOS} para cambios fundamentales (reducción de derechos, cambio de fin del fideicomiso, etc.).
  \item Con consentimiento de \textbf{75\% de FIDEICOMISARIOS} para cambios administrativos (reglamento, cuotas, procedimientos).
  \item Las modificaciones deben formalizarse ante notario e inscribirse en Registro Público.
\end{enumerate}

\SubClause{Artículo 30. Integridad del Documento}

Las partes reconocen que este contrato, junto con el Reglamento Interno (Anexo A), constituye el acuerdo completo y final entre ellas. Previos acuerdos verbales o escritos quedan sin efecto.

\SubClause{Artículo 31. Gastos y Costos}

Todos los gastos derivados de:
\begin{itemize}
  \item Elaboración de este contrato
  \item Firma ante notario
  \item Inscripción en Registro Público
  \item Trámites iniciales de constitución
\end{itemize}

Son asumidos \textbf{CONJUNTAMENTE} por los FIDEICOMISARIOS en partes iguales (25\% cada uno).

Gastos futuros de administración son prorrateados entre todos según participación.

\SubClause{Artículo 32. Notificaciones}

Cualquier notificación requerida bajo este contrato se realiza:

\begin{enumerate}[label=\arabic*.]
  \item Por correo certificado con acuse de recibo a domicilio registrado.
  \item Por correo electrónico certificado (con confirmación de lectura).
  \item Por entrega personal con recibo firmado.
\end{enumerate}

Se considera notificación válida 48 horas después de enviada.

\newpage
