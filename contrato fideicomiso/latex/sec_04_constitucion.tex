% === SECCIÓN 4: CONSTITUCIÓN DEL FIDEICOMISO ===

\Clause{Constitución del Fideicomiso}

\SubClause{Artículo 3. Voluntad de las Partes}

Las partes manifiestan su voluntad de constituir este FIDEICOMISO INMOBILIARIO conforme a la Ley General de Títulos y Operaciones de Crédito, al Código Civil Federal aplicable, a la Ley de Inversión Extranjera (en su caso), y a la NOM-029-SE-2021 ``Prácticas comerciales—Requisitos informativos para la prestación del servicio de tiempo compartido''.

\SubClause{Artículo 4. Efectos y Validez}

\begin{enumerate}[label=\arabic*.]
  \item El presente fideicomiso surte efecto a partir de la inscripción en el Registro Público de la Propiedad de la Sección de Propiedad de Mérida, Yucatán.
  
  \item El fideicomiso tiene carácter irrevocable por la voluntad unilateral del FIDEICOMITENTE, requiriendo para cualquier modificación sustancial el consentimiento de al menos tres (3) de los cuatro (4) FIDEICOMISARIOS.
  
  \item El fideicomiso tendrá vigencia de \textbf{CINCUENTA (50) AÑOS}, conforme a lo permitido por la Ley de Inversión Extranjera. Al vencimiento de este plazo, los derechos fiduciarios se consolidarán en dominio directo de los FIDEICOMISARIOS en proporción a su participación (1/4 cada uno), o se constituirá un nuevo fideicomiso de común acuerdo.
\end{enumerate}

\SubClause{Artículo 5. Transmisión de Dominio}

Por este acto, el FIDEICOMITENTE transmite al FIDUCIARIO la propiedad del inmueble descrito en el Artículo 2, quien la mantiene en administración a favor de los FIDEICOMISARIOS conforme a este contrato.

El FIDEICOMITENTE se reserva:

\begin{itemize}
  \item El derecho a que le sean pagados los importes que correspondan por concepto de aportación del inmueble (si aplica).
  \item Ningún otro derecho real sobre el inmueble una vez aportado.
\end{itemize}

\newpage
