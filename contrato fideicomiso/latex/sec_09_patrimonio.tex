% === SECCIÓN 9: PATRIMONIO Y FINANCIAMIENTO ===

\Clause{Patrimonio y Financiamiento}

\SubClause{Artículo 18. Patrimonio del Fideicomiso}

El patrimonio del FIDEICOMISO está constituido por:

\begin{enumerate}[label=\arabic*.]
  \item \textbf{Bienes inmuebles:} El departamento de 100 m² descrito en Artículo 2.
  \item \textbf{Bienes muebles:} Muebles, electrodomésticos, equipamiento y decoración incluidos en el inmueble.
  \item \textbf{Derechos incorporales:} Marcas, nombres comerciales, plataformas de reservación, dominios web asociados al proyecto.
  \item \textbf{Fondos de reserva:} Capital acumulado destinado a mantenimiento y contingencias.
  \item \textbf{Ingresos derivados:} Rentas, depósitos de seguridad, ingresos por servicios adicionales.
\end{enumerate}

\SubClause{Artículo 19. Fondo de Reserva}

\textbf{19.1 Constitución}

Se constituye un \textbf{FONDO DE RESERVA} capitalizado con:
\begin{itemize}
  \item \textbf{5\% de ingresos netos anuales} dirigidos a este fondo.
  \item Monto objetivo: Equivalente a \textbf{6 meses de gastos operativos}.
  \item Proyectado: Aproximadamente \$150,000 MXN en condiciones normales.
\end{itemize}

\textbf{19.2 Destino}

El Fondo de Reserva se utiliza exclusivamente para:
\begin{enumerate}[label=\arabic*.]
  \item Reparaciones estructurales inesperadas.
  \item Sustitución de equipos (aire acondicionado, calentador, refrigerador).
  \item Períodos prolongados de baja ocupación.
  \item Contingencias no cubiertas por seguros.
\end{enumerate}

\textbf{19.3 Administración}
\begin{itemize}
  \item Mantenido en cuenta de depósito en banco de primera línea.
  \item Retiros requieren aprobación de 75\% de FIDEICOMISARIOS.
  \item Informe anual de movimientos.
\end{itemize}

\SubClause{Artículo 20. Ingresos y Gastos}

\textbf{20.1 Ingresos del Fideicomiso}
\begin{itemize}
  \item Rentas por ocupación de huéspedes.
  \item Depósitos de seguridad retenidos por daños.
  \item Servicios adicionales vendidos (late checkout, cuna, etc.).
  \item Intereses sobre depósitos en fondo de reserva.
\end{itemize}

\textbf{20.2 Gastos del Fideicomiso}
\begin{itemize}
  \item Mantenimiento y reparaciones.
  \item Servicios (agua, luz, internet, gas).
  \item Seguros integral.
  \item Limpieza y servicios de personal.
  \item Comisión de Empresa Gestora.
  \item Impuestos prediales y derechos municipales.
  \item Comisión del FIDUCIARIO.
  \item Inscripciones registrales y notariales.
\end{itemize}

\textbf{20.3 Presupuesto anual}

Se elabora presupuesto proyectado antes del 30 de noviembre de cada año, aprobado en asamblea ordinaria, detallando:
\begin{itemize}
  \item Ingresos proyectados por escenario.
  \item Gastos estimados por categoría.
  \item Cálculo de cuota anual por FIDEICOMISARIO.
  \item Política de distribución de rendimientos.
\end{itemize}

\SubClause{Artículo 21. Seguros e Indemnización}

\textbf{21.1 Cobertura de seguros}

El FIDUCIARIO contrata seguros integrales que cubren:

\begin{enumerate}[label=\arabic*.]
  \item \textbf{Seguro de daños físicos:} Incendio, explosión, rayo, hurto, robo (cobertura mínima: 100\% del valor avalúo).
  \item \textbf{Seguro de responsabilidad civil:} Daños a terceros por accidentes en la propiedad.
  \item \textbf{Seguro de contenidos:} Equipos, muebles, decoración.
  \item \textbf{Seguro de pérdida de rentas:} En caso de daño que impida rentabilidad.
\end{enumerate}

Primas: Incluidas en gastos operativos, prorrateadas entre FIDEICOMISARIOS.

\textbf{21.2 Indemnizaciones}

En caso de siniestro:
\begin{enumerate}[label=\arabic*.]
  \item El FIDUCIARIO gestiona la reclamación ante asegurador.
  \item Fondos se utilizan para reparación/restitución.
  \item Si hay excedente, se distribuye entre FIDEICOMISARIOS.
  \item Si hay déficit, se cubre con Fondo de Reserva o aportación extraordinaria.
\end{enumerate}

\newpage
