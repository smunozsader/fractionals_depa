% === SECCIÓN 6: SISTEMA DE ASIGNACIÓN DE SEMANAS (ROTACIÓN SERPIENTE) ===

\Clause{Sistema de Asignación de Semanas (Rotación Serpiente)}

\SubClause{Artículo 9. Método de Rotación Serpiente (Zig-Zag)}

La asignación de las 52 semanas anuales entre los cuatro FIDEICOMISARIOS se rige por el \textbf{SISTEMA DE ROTACIÓN SERPIENTE}, denominado también ``draft zig-zag'', que garantiza equidad matemática perfecta en la distribución de temporadas.

\textbf{Fundamento del sistema:}

Este sistema resuelve la inequidad inherente a otros métodos (rotación simple, sorteo aleatorio) al invertir el orden de selección en cada ronda, asegurando que quien elige último en una ronda elije primero en la siguiente.

\textbf{Estructura de temporadas:}

Las 52 semanas se clasifican en tres categorías según demanda turística y condiciones climáticas:

\begin{enumerate}[label=\arabic*.]
  \item \textbf{Temporada ALTA (26 semanas):} Diciembre-marzo y julio-agosto. Clima óptimo (18-28°C), festividades importantes (Navidad, Año Nuevo, Semana Santa, vacaciones escolares). Tarifa promedio \$2,500 MXN/noche.
  
  \item \textbf{Temporada MEDIA (14 semanas):} Abril, mayo parcial, octubre-noviembre. Transición climática, eventos culturales (Carnaval, festivales locales). Tarifa promedio \$1,800 MXN/noche.
  
  \item \textbf{Temporada BAJA (12 semanas):} Mayo tardío, junio, septiembre. Temporada de lluvias (35°C+, precipitaciones vespertinas). Menores multitudes, precios accesibles. Tarifa promedio \$1,200 MXN/noche.
\end{enumerate}

\SubClause{Artículo 10. Proceso de Asignación Anual}

\textbf{Fase 1: Preparación (Pre-Draft)}

\textit{Paso 1: Determinación de orden inicial mediante sorteo}

Antes del primer proceso de selección, se sortea el orden permanente de los FIDEICOMISARIOS mediante método aleatorio neutral:

\begin{itemize}
  \item Se escriben los nombres de los cuatro (4) FIDEICOMISARIOS en papeles cerrados.
  \item Un notario extrae los papeles en orden.
  \item Primer papel: Posición ``A'' (elige primero en ronda normal)
  \item Segundo papel: Posición ``B'' (elige segundo)
  \item Tercer papel: Posición ``C'' (elige tercero)
  \item Cuarto papel: Posición ``D'' (elige cuarto en ronda normal)
\end{itemize}

\textbf{Criticidad:} Las posiciones A, B, C, D son \textbf{PERMANENTES} para todos los años subsecuentes. El sistema serpiente (no la rotación de letras) genera la equidad.

\textit{Paso 2: Clasificación de las 52 semanas}

Se publica el calendario anual clasificando todas las semanas por temporada:
\begin{itemize}
  \item Semanas 1-26: Temporada ALTA (orden de prioridad según demanda)
  \item Semanas 27-40: Temporada MEDIA
  \item Semanas 41-52: Temporada BAJA
\end{itemize}

\textit{Paso 3: Establecimiento de fecha y modalidad del Draft}

Se agenda una reunión con todos los FIDEICOMISARIOS para ejecutar el proceso de selección voluntaria. Modalidades permitidas:
\begin{itemize}
  \item Presencial: En las oficinas del FIDUCIARIO o notaría
  \item Virtual: Videoconferencia grabada en plataforma segura
\end{itemize}

Requisito indispensable: Quórum de los 4 FIDEICOMISARIOS (o representantes legales con poder notarial).

\textbf{Fase 2: Ejecución del Draft (Selección Voluntaria)}

El Draft se ejecuta en 13 rondas (52 semanas $\div$ 4 tickets = 13 rondas por fideicomisario).

\textbf{Ronda 1 (Temporada ALTA):} Orden A $\rightarrow$ B $\rightarrow$ C $\rightarrow$ D

\textbf{Ronda 2 (Temporada ALTA):} Orden INVERTIDO: D $\rightarrow$ C $\rightarrow$ B $\rightarrow$ A $\leftarrow$ \textit{¡SERPIENTE ACTIVADA!}

\textbf{Ronda 3 (Temporada ALTA):} Orden normal nuevamente: A $\rightarrow$ B $\rightarrow$ C $\rightarrow$ D

\textbf{Patrón continúa:} Alterno normal/invertido hasta completar todas las temporadas.

\textbf{Resultado equilibrado:}
\begin{itemize}
  \item Cada FIDEICOMISARIO obtiene 13 semanas (91 días anuales)
  \item Ticket A: $\sim$7 semanas ALTA + 3 MEDIA + 3 BAJA
  \item Ticket B: $\sim$7 semanas ALTA + 3 MEDIA + 3 BAJA
  \item Ticket C: $\sim$6 semanas ALTA + 4 MEDIA + 3 BAJA
  \item Ticket D: $\sim$6 semanas ALTA + 4 MEDIA + 3 BAJA
\end{itemize}

Diferencia máxima: 1 semana ALTA (matemáticamente inevitable al dividir 26 entre 4), compensada con 1 semana MEDIA adicional.

\textbf{Fase 3: Documentación y Cierre}

\begin{enumerate}[label=\arabic*.]
  \item Al finalizar, el facilitador lee en voz alta el calendario completo de cada FIDEICOMISARIO.
  \item Se solicita confirmación verbal de aceptación de los cuatro FIDEICOMISARIOS.
  \item Se genera acta notarial con la asignación definitiva y firmas de los participantes.
  \item Se entrega una copia del calendario a cada FIDEICOMISARIO en papel membretado.
  \item Se publica en la plataforma de administración del fideicomiso dentro de 48 horas.
\end{enumerate}

\SubClause{Artículo 11. Reglas y Casos Especiales}

\textbf{Límite de tiempo por selección:} Cada FIDEICOMISARIO dispone de máximo 3 minutos para elegir. Transcurrido ese tiempo, pasa al siguiente fideicomisario y realiza su selección al finalizar la ronda.

\textbf{Ausencia de un FIDEICOMISARIO:}
\begin{itemize}
  \item Debe nombrar un representante legal con poder notarial, O
  \item Enviar lista de preferencias ordenada (1ª opción, 2ª opción, 3ª opción, etc.)
  \item El facilitador selecciona automáticamente según lista en su turno
\end{itemize}

\textbf{Imposibilidad de celebrar Draft:} Si falta alguno de los cuatro FIDEICOMISARIOS sin representación ni lista de preferencias, el Draft se suspende y se reprograma con todos presentes.

\textbf{Cambios después del Draft:} Las asignaciones son definitivas por año calendario. No se permiten reclamos posteriores. SÍ se permiten intercambios entre fideicomisarios con acuerdo mutuo documentado.

\textbf{Desempates:} Si dos FIDEICOMISARIOS desean la misma semana simultáneamente, tiene prioridad quien está en turno según el orden serpiente.

\SubClause{Artículo 12. Modificación del Sistema}

El Sistema de Rotación Serpiente puede modificarse únicamente con consentimiento de al menos TRES (3) de los CUATRO (4) FIDEICOMISARIOS, mediante asamblea extraordinaria. Cualquier modificación debe ser documentada ante notario e inscrita en el Registro Público.

\newpage
