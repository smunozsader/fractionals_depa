% ============================================================================
% ANEXO: PROCEDIMIENTO DE SELECCIÓN VOLUNTARIA DE SEMANAS
% Sistema de Rotación Serpiente con Draft de Selección
% Propiedad Fraccional - Mérida, Yucatán
% ============================================================================
\documentclass[11pt,letterpaper,oneside]{scrartcl}

% ============================================================================
% PAQUETES ESENCIALES
% ============================================================================
\usepackage[utf8]{inputenc}
\usepackage[spanish,mexico]{babel}
\usepackage[T1]{fontenc}
\usepackage{lmodern}

% Matemáticas y símbolos
\usepackage{amsmath,amssymb}

% Geometría y márgenes
\usepackage[letterpaper,margin=2.5cm,top=2cm,bottom=2.5cm]{geometry}

% Gráficos y color
\usepackage{graphicx}
\usepackage{xcolor}
\definecolor{turquesa}{RGB}{0,168,150}
\definecolor{verdeoscuro}{RGB}{0,100,80}
\definecolor{gris}{RGB}{64,64,64}
\definecolor{grisclaro}{RGB}{245,245,245}
\definecolor{naranja}{RGB}{255,140,0}
\definecolor{rojo}{RGB}{220,50,50}
\definecolor{amarillo}{RGB}{230,180,50}
\definecolor{azulclaro}{RGB}{100,180,220}

% Tablas profesionales
\usepackage{booktabs}
\usepackage{tabularx}
\usepackage{multirow}
\usepackage{array}

% Enlaces e hipervínculos
\usepackage[hidelinks,colorlinks=true,linkcolor=verdeoscuro,urlcolor=turquesa]{hyperref}

% Cajas y destacados
\makeatletter
\providecommand\NewStructureName[1]{}
\providecommand\AssignStructureRole[2]{}
\makeatother
\usepackage{tcolorbox}
\tcbuselibrary{skins,breakable}

% Encabezados y pies de página
\usepackage{scrlayer-scrpage}
\clearpairofpagestyles
\ihead{\color{gris}\small Anexo - Procedimiento de Selección}
\ohead{\color{gris}\small \today}
\cfoot{\pagemark}
\setkomafont{pagehead}{\small\normalfont}
\setkomafont{pagefoot}{\small\normalfont}

% Formato de secciones
\addtokomafont{section}{\color{verdeoscuro}}
\addtokomafont{subsection}{\color{turquesa}}

% Listas personalizadas
\usepackage{enumitem}
\setlist[itemize]{leftmargin=*,labelsep=0.5em,itemsep=0.3em}
\setlist[enumerate]{leftmargin=*,labelsep=0.5em,itemsep=0.3em}

% Formato de moneda
\newcommand{\MXN}[1]{\$#1~MXN}

% ============================================================================
% CAJAS PERSONALIZADAS
% ============================================================================
\newtcolorbox{cajadestacado}[1]{
    colback=grisclaro,
    colframe=verdeoscuro,
    fonttitle=\bfseries\color{white},
    coltitle=white,
    colbacktitle=verdeoscuro,
    title=#1,
    enhanced,
    attach boxed title to top left={yshift=-2mm,xshift=5mm},
    boxed title style={sharp corners}
}

\newtcolorbox{cajaalerta}{
    colback=white,
    colframe=naranja,
    fonttitle=\bfseries\color{white},
    coltitle=white,
    colbacktitle=naranja,
    title=Importante,
    enhanced,
    attach boxed title to top left={yshift=-2mm,xshift=5mm},
    boxed title style={sharp corners}
}

\newtcolorbox{cajaejemplo}{
    colback=azulclaro!10,
    colframe=azulclaro,
    fonttitle=\bfseries\color{white},
    coltitle=white,
    colbacktitle=azulclaro,
    title=Ejemplo Práctico,
    enhanced,
    breakable,
    attach boxed title to top left={yshift=-2mm,xshift=5mm},
    boxed title style={sharp corners}
}

% ============================================================================
% DOCUMENTO
% ============================================================================
\begin{document}

% ----------------------------------------------------------------------------
% PORTADA
% ----------------------------------------------------------------------------
\begin{titlepage}
    \centering
    \vspace*{2cm}
    
    {\Huge\bfseries\color{verdeoscuro} ANEXO TÉCNICO}
    
    \vspace{0.8cm}
    
    {\LARGE\color{turquesa} Procedimiento de Selección\\[0.3cm] Voluntaria de Semanas}
    
    \vspace{0.5cm}
    
    {\Large Sistema de Rotación Serpiente\\[0.2cm] Modalidad Draft con Orden Equitativo}
    
    \vspace{1.5cm}
    
    \begin{tcolorbox}[colback=grisclaro,colframe=turquesa,width=0.8\textwidth,arc=0mm]
        \centering
        \textbf{\large Datos del Proyecto}\\[0.4cm]
        \begin{tabular}{rl}
            \textbf{Propiedad:} & Departamento Mérida, Yucatán \\[0.15cm]
            \textbf{Tickets:} & 4 copropietarios \\[0.15cm]
            \textbf{Semanas:} & 52 semanas/año \\[0.15cm]
            \textbf{Derechos por ticket:} & 13 semanas/año \\
        \end{tabular}
    \end{tcolorbox}
    
    \vspace{2cm}
    
    {\color{gris}
    \textit{Documento complementario al}\\[0.2cm]
    {\large\bfseries Reporte Principal de Propiedad Fraccional}
    }
    
    \vfill
    
    {\large\color{gris} \today}
    
\end{titlepage}

% ----------------------------------------------------------------------------
% TABLA DE CONTENIDOS
% ----------------------------------------------------------------------------
\tableofcontents
\newpage

% ============================================================================
% INTRODUCCIÓN
% ============================================================================
\section{Introducción}

\subsection{Objetivo del Documento}

Este anexo técnico presenta el \textbf{procedimiento detallado} para la selección voluntaria de semanas bajo el Sistema de Rotación Serpiente. A diferencia de una asignación automática pre-calculada, este método permite que los copropietarios elijan activamente sus semanas preferidas respetando el orden equitativo del sistema serpiente.

\subsection{Dos Enfoques Posibles}

Existen dos formas de implementar el Sistema de Rotación Serpiente:

\begin{enumerate}
    \item \textbf{Asignación Automática}: Un algoritmo pre-calcula las 13 semanas de cada ticket y las asigna sin intervención de los propietarios. Es totalmente objetivo pero menos flexible.
    
    \item \textbf{Selección Voluntaria (Draft)}: Los propietarios eligen sus semanas siguiendo el orden serpiente en un proceso tipo ``borrador deportivo''. Respeta preferencias personales y genera mayor satisfacción.
\end{enumerate}

\begin{cajadestacado}{Enfoque Recomendado}
Este documento describe el \textbf{Enfoque 2: Selección Voluntaria}, que combina la equidad matemática del sistema serpiente con la flexibilidad de permitir que los copropietarios expresen sus preferencias individuales.
\end{cajadestacado}

% ============================================================================
% FUNDAMENTOS DEL SISTEMA
% ============================================================================
\section{Fundamentos del Sistema Serpiente}

\subsection{Principio de Equidad}

El sistema serpiente garantiza que ningún copropietario tenga ventaja permanente al \textbf{invertir el orden de selección} en cada ronda:

\begin{itemize}
    \item \textbf{Ronda 1}: A $\rightarrow$ B $\rightarrow$ C $\rightarrow$ D (orden normal)
    \item \textbf{Ronda 2}: D $\rightarrow$ C $\rightarrow$ B $\rightarrow$ A (orden invertido)
    \item \textbf{Ronda 3}: A $\rightarrow$ B $\rightarrow$ C $\rightarrow$ D (orden normal)
    \item \textbf{Ronda 4}: D $\rightarrow$ C $\rightarrow$ B $\rightarrow$ A (orden invertido)
\end{itemize}

Este patrón continúa hasta completar las 13 rondas necesarias (52 semanas $\div$ 4 tickets = 13 rondas).

\begin{cajadestacado}{Recordatorio Importante}
\textbf{A, B, C, D} son solo etiquetas de posición que rotan cada año. El copropietario que fue ``A'' en 2026 será ``D'' en 2027, ``C'' en 2028 y ``B'' en 2029. Todos pasan por todas las posiciones en un ciclo de 4 años.
\end{cajadestacado}

\subsection{Ventajas de la Selección Voluntaria}

La modalidad de selección voluntaria ofrece beneficios significativos:

\begin{itemize}
    \item \textbf{Respeto a preferencias}: Cada propietario elige según sus necesidades familiares, laborales o de viaje
    \item \textbf{Mayor satisfacción}: Al participar activamente, hay mayor aceptación del resultado
    \item \textbf{Transparencia total}: Todos observan el proceso y entienden por qué cada semana fue asignada
    \item \textbf{Flexibilidad ante cambios}: Si las circunstancias familiares cambian, el proceso se adapta naturalmente
\end{itemize}

% ============================================================================
% PROCEDIMIENTO PASO A PASO
% ============================================================================
\section{Procedimiento de Selección: Paso a Paso}

\subsection{Fase 1: Preparación (Pre-Draft)}

\subsubsection{1.1 Determinar el Orden Inicial mediante Sorteo}

\textbf{MUY IMPORTANTE}: Las letras A, B, C y D son \textbf{únicamente etiquetas} para identificar posiciones en el orden de selección. \textbf{NO representan personas específicas ni derechos permanentes}.

Antes de comenzar la selección del primer año, se debe sortear \textbf{aleatoriamente} quién ocupará cada posición. Tres métodos justos para el sorteo:

\begin{enumerate}
    \item \textbf{Sorteo aleatorio con urna} (RECOMENDADO): Se escriben los 4 nombres en papeles, se colocan en una urna y se extraen uno por uno. El primer nombre extraído = Posición A, segundo = Posición B, tercero = Posición C, cuarto = Posición D.
    
    \item \textbf{Por orden cronológico de compra}: Quien firmó su ticket primero = Posición A, segundo = Posición B, etc. Este método solo es justo si las compras ocurrieron en días diferentes.
    
    \item \textbf{Por acuerdo mutuo documentado}: Los 4 copropietarios negocian y deciden voluntariamente, registrando el acuerdo por escrito.
\end{enumerate}

\begin{cajaalerta}
\textbf{CRÍTICO - Sistema de Rotación Anual}

El orden inicial es \textbf{TEMPORAL} y se sortea solo UNA VEZ. A partir del segundo año, el orden ROTA automáticamente para que todos pasen por todas las posiciones:

\vspace{0.3cm}

\textbf{Ejemplo}: Si en el sorteo inicial quedó:
\begin{itemize}
    \item Jorge = Posición A
    \item María = Posición B
    \item Roberto = Posición C
    \item Ana = Posición D
\end{itemize}

\textbf{Rotación en años siguientes}:
\begin{itemize}
    \item \textbf{Año 1 (2026)}: Jorge (A) $\rightarrow$ María (B) $\rightarrow$ Roberto (C) $\rightarrow$ Ana (D)
    \item \textbf{Año 2 (2027)}: María (A) $\rightarrow$ Roberto (B) $\rightarrow$ Ana (C) $\rightarrow$ Jorge (D)
    \item \textbf{Año 3 (2028)}: Roberto (A) $\rightarrow$ Ana (B) $\rightarrow$ Jorge (C) $\rightarrow$ María (D)
    \item \textbf{Año 4 (2029)}: Ana (A) $\rightarrow$ Jorge (B) $\rightarrow$ María (C) $\rightarrow$ Roberto (D)
    \item \textbf{Año 5 (2030)}: Jorge (A) nuevamente - reinicia el ciclo
\end{itemize}

\textbf{Resultado}: En 4 años, todos pasan por las 4 posiciones exactamente 1 vez. Equidad total a largo plazo.
\end{cajaalerta}

\subsubsection{1.2 Clasificar las 52 Semanas}

Antes del Draft, se debe publicar la clasificación de todas las semanas por temporada:

\begin{table}[h]
\centering
\begin{tabular}{lcc}
\toprule
\textbf{Temporada} & \textbf{Semanas} & \textbf{Meses Aproximados} \\
\midrule
\textbf{ALTA} & 26 semanas & Dic-Mar, Jul-Ago \\
\textbf{MEDIA} & 14 semanas & Abr-May, Oct-Nov \\
\textbf{BAJA} & 12 semanas & Jun, Sep \\
\bottomrule
\end{tabular}
\caption{Clasificación de temporadas}
\end{table}

Esta información debe estar disponible para todos los copropietarios \textbf{al menos 30 días antes} del proceso de selección.

\subsubsection{1.3 Establecer Fecha y Modalidad del Draft}

El proceso de selección puede realizarse de dos formas:

\begin{itemize}
    \item \textbf{Presencial}: Reunión de los 4 copropietarios con notario/fiduciario
    \item \textbf{Virtual}: Videoconferencia grabada con participación simultánea
\end{itemize}

\textbf{Requisitos mínimos}:
\begin{itemize}
    \item Quórum: 4 de 4 propietarios presentes
    \item Registro: Audio/video o acta notarial
    \item Tiempo estimado: 2-3 horas
    \item Facilitador: Notario, fiduciario o gestor neutral
\end{itemize}

\subsection{Fase 2: Ejecución del Draft}

\subsubsection{2.1 Ronda 1: Temporada ALTA (Semanas 1-4)}

El facilitador inicia el proceso anunciando:

\begin{quote}
\textit{``Ronda 1 de Temporada ALTA. Orden: A $\rightarrow$ B $\rightarrow$ C $\rightarrow$ D. Disponibles: 26 semanas de temporada alta.''}
\end{quote}

\textbf{Turno 1.1 - Ticket A elige primero}:
\begin{itemize}
    \item El facilitador pregunta: \textit{``Ticket A, ¿cuál semana de temporada ALTA deseas?''}
    \item Ticket A responde, por ejemplo: \textit{``Semana 52 (última de diciembre)''}
    \item El facilitador registra y elimina esa semana del inventario disponible
\end{itemize}

\textbf{Turno 1.2 - Ticket B elige segundo}:
\begin{itemize}
    \item \textit{``Ticket B, elige de las 25 semanas ALTA restantes''}
    \item Ticket B: \textit{``Semana 14 (Semana Santa)''}
\end{itemize}

\textbf{Turno 1.3 - Ticket C elige tercero}:
\begin{itemize}
    \item \textit{``Ticket C, elige de las 24 semanas ALTA restantes''}
    \item Ticket C: \textit{``Semana 30 (julio)''}
\end{itemize}

\textbf{Turno 1.4 - Ticket D elige último}:
\begin{itemize}
    \item \textit{``Ticket D, elige de las 23 semanas ALTA restantes''}
    \item Ticket D: \textit{``Semana 1 (primera de enero)''}
\end{itemize}

\subsubsection{2.2 Ronda 2: Temporada ALTA (Semanas 5-8) - ORDEN INVERTIDO}

El facilitador anuncia:

\begin{quote}
\textit{``Ronda 2 de Temporada ALTA. Orden INVERTIDO: D $\rightarrow$ C $\rightarrow$ B $\rightarrow$ A. Disponibles: 22 semanas de temporada alta.''}
\end{quote}

\textbf{Turno 2.1 - Ticket D elige primero} (compensación por haber elegido último en Ronda 1):
\begin{itemize}
    \item Ticket D: \textit{``Semana 31 (segunda de agosto)''}
\end{itemize}

\textbf{Turno 2.2 - Ticket C elige segundo}:
\begin{itemize}
    \item Ticket C: \textit{``Semana 8 (febrero)''}
\end{itemize}

\textbf{Turno 2.3 - Ticket B elige tercero}:
\begin{itemize}
    \item Ticket B: \textit{``Semana 51 (Navidad)''}
\end{itemize}

\textbf{Turno 2.4 - Ticket A elige último}:
\begin{itemize}
    \item Ticket A: \textit{``Semana 29 (julio)''}
\end{itemize}

\subsubsection{2.3 Rondas 3-7: Completar Temporada ALTA}

Se continúa alternando el orden (normal/invertido) hasta agotar las 26 semanas de temporada alta:

\begin{itemize}
    \item \textbf{Ronda 3}: A $\rightarrow$ B $\rightarrow$ C $\rightarrow$ D (semanas 9-12 ALTA)
    \item \textbf{Ronda 4}: D $\rightarrow$ C $\rightarrow$ B $\rightarrow$ A (semanas 13-16 ALTA)
    \item \textbf{Ronda 5}: A $\rightarrow$ B $\rightarrow$ C $\rightarrow$ D (semanas 17-20 ALTA)
    \item \textbf{Ronda 6}: D $\rightarrow$ C $\rightarrow$ B $\rightarrow$ A (semanas 21-24 ALTA)
    \item \textbf{Ronda 7}: A $\rightarrow$ B (semanas 25-26 ALTA - solo 2 semanas restantes)
\end{itemize}

\textbf{Resultado al finalizar temporada ALTA}:
\begin{itemize}
    \item Tickets A y B: 7 semanas ALTA cada uno
    \item Tickets C y D: 6 semanas ALTA cada uno
\end{itemize}

\subsubsection{2.4 Rondas 8-10: Temporada MEDIA (14 semanas)}

Se inicia una nueva fase con las 14 semanas de temporada MEDIA, continuando el patrón serpiente:

\begin{itemize}
    \item \textbf{Ronda 8}: A $\rightarrow$ B $\rightarrow$ C $\rightarrow$ D (semanas 1-4 MEDIA)
    \item \textbf{Ronda 9}: D $\rightarrow$ C $\rightarrow$ B $\rightarrow$ A (semanas 5-8 MEDIA)
    \item \textbf{Ronda 10}: A $\rightarrow$ B $\rightarrow$ C $\rightarrow$ D (semanas 9-12 MEDIA)
    \item \textbf{Ronda 11}: D $\rightarrow$ C (semanas 13-14 MEDIA - solo 2 semanas restantes)
\end{itemize}

\textbf{Resultado al finalizar temporada MEDIA}:
\begin{itemize}
    \item Tickets A y B: 3 semanas MEDIA cada uno
    \item Tickets C y D: 4 semanas MEDIA cada uno
\end{itemize}

\subsubsection{2.5 Rondas 11-13: Temporada BAJA (12 semanas)}

Finalmente, se seleccionan las 12 semanas de temporada BAJA:

\begin{itemize}
    \item \textbf{Ronda 11} (continuación): B $\rightarrow$ A (semanas 1-2 BAJA)
    \item \textbf{Ronda 12}: D $\rightarrow$ C $\rightarrow$ B $\rightarrow$ A (semanas 3-6 BAJA)
    \item \textbf{Ronda 13}: A $\rightarrow$ B $\rightarrow$ C $\rightarrow$ D (semanas 7-10 BAJA)
    \item \textbf{Ronda 14}: D $\rightarrow$ C (semanas 11-12 BAJA - últimas 2 semanas)
\end{itemize}

\textbf{Resultado final}:
\begin{itemize}
    \item Todos los tickets: 3 semanas BAJA cada uno
\end{itemize}

\subsection{Fase 3: Documentación y Cierre}

\subsubsection{3.1 Registro Oficial}

Al finalizar el proceso, el facilitador debe:

\begin{enumerate}
    \item Leer en voz alta el calendario completo de cada ticket
    \item Solicitar confirmación verbal de los 4 copropietarios
    \item Documentar en acta notarial o grabación
    \item Generar 4 copias del calendario final (una por propietario)
\end{enumerate}

\subsubsection{3.2 Publicación del Calendario}

Dentro de las 48 horas siguientes al Draft, se debe:

\begin{itemize}
    \item Enviar calendario digital a todos los copropietarios
    \item Publicar en plataforma de gestión del fideicomiso
    \item Archivar en expediente legal del proyecto
    \item Notificar a la empresa gestora de la propiedad
\end{itemize}

% ============================================================================
% REGLAS Y CASOS ESPECIALES
% ============================================================================
\section{Reglas y Casos Especiales}

\subsection{Límite de Tiempo por Selección}

Para mantener fluidez en el proceso:

\begin{itemize}
    \item Cada propietario tiene \textbf{máximo 3 minutos} para elegir
    \item Si no decide en ese tiempo, el facilitador pasa al siguiente
    \item El propietario que no eligió selecciona al final de la ronda
\end{itemize}

\subsection{Ausencia de un Copropietario}

Si un propietario no puede asistir al Draft:

\begin{itemize}
    \item Debe nombrar un \textbf{representante legal} con poder notarial
    \item Puede enviar lista de preferencias por escrito (1ª opción, 2ª opción, etc.)
    \item El facilitador selecciona automáticamente según la lista en su turno
\end{itemize}

\begin{cajaalerta}
Si un propietario no asiste ni envía representante, el Draft \textbf{NO puede realizarse}. Se requiere participación de los 4 tickets obligatoriamente.
\end{cajaalerta}

\subsection{Cambios Después del Draft}

Una vez concluido el proceso de selección:

\begin{itemize}
    \item Las asignaciones son \textbf{FINALES} para ese año calendario
    \item No se permiten reclamaciones posteriores
    \item Los intercambios entre propietarios SÍ están permitidos con acuerdo mutuo documentado
\end{itemize}

\subsection{Desempates}

En caso de que dos propietarios deseen la misma semana simultáneamente:

\begin{itemize}
    \item Tiene prioridad quien está en turno según el orden serpiente
    \item No hay excepciones ni negociaciones durante el Draft
    \item El propietario que perdió prioridad elige otra semana disponible
\end{itemize}

% ============================================================================
% EJEMPLO COMPLETO
% ============================================================================
\section{Ejemplo Completo de Draft}

\begin{cajaejemplo}
\textbf{Escenario}: Cuatro amigos compraron una propiedad fraccional en Mérida. Reunión: 1 nov 2025.

\vspace{0.3cm}

\textbf{PASO 1: Sorteo del Orden Inicial}

El notario extrae papeles de urna:
\begin{enumerate}[itemsep=0.1em]
    \item 1er papel: Jorge $\rightarrow$ \textbf{Posición A (2026)}
    \item 2do papel: María $\rightarrow$ \textbf{Posición B (2026)}
    \item 3er papel: Roberto $\rightarrow$ \textbf{Posición C (2026)}
    \item 4to papel: Ana $\rightarrow$ \textbf{Posición D (2026)}
\end{enumerate}

\textit{Este sorteo se hace UNA SOLA VEZ. Años siguientes: rotación automática.}

\vspace{0.3cm}

\textbf{PASO 2: Draft de Selección (Año 2026)}

\textbf{Perfiles}:
Jorge (A): familia, hijos en primaria | María (B): soltera, flexible | Roberto (C): jubilado | Ana (D): pareja, verano

\vspace{0.3cm}

\textbf{Ronda 1 - ALTA}:
Jorge(A): S15-Semana Santa | María(B): S52-Año Nuevo | Roberto(C): S8-Febrero | Ana(D): S31-Agosto

\textbf{Ronda 2 - ALTA (invertida)}:
Ana(D): S30-Julio | Roberto(C): S9-Febrero | María(B): S51-Navidad | Jorge(A): S14-Marzo

\textit{...continúa proceso hasta 13 semanas cada uno...}

\vspace{0.3cm}

\textbf{Resultado 2026 - Jorge}: 7 sem ALTA (2 Semana Santa + 3 verano + 2 dic) | 3 sem MEDIA (abr-nov) | 3 sem BAJA (jun)

\vspace{0.3cm}

\textbf{¿Y en 2027?} Rotación automática: María=A | Roberto=B | Ana=C | Jorge=D

Jorge pasa de primera a última posición. En 2028=C, 2029=B. \textbf{Ciclo completo en 4 años}.

\vspace{0.2cm}

\textbf{Conclusión}: Sorteo justo inicial + rotación anual = equidad perfecta a largo plazo.
\end{cajaejemplo}

% ============================================================================
% VENTAJAS Y DESVENTAJAS
% ============================================================================
\section{Comparación: Selección Voluntaria vs. Asignación Automática}

\begin{table}[htbp]
\centering
\footnotesize
\begin{tabularx}{\textwidth}{X>{\centering\arraybackslash}p{3.2cm}>{\centering\arraybackslash}p{3.2cm}}
\toprule
\textbf{Criterio} & \textbf{Automática} & \textbf{Voluntaria} \\
\midrule
Equidad matemática & $\bullet\bullet\bullet\bullet\bullet$ & $\bullet\bullet\bullet\bullet\bullet$ \\
Satisfacción copropietarios & $\bullet\bullet\bullet$ & $\bullet\bullet\bullet\bullet\bullet$ \\
Tiempo implementación & $\bullet\bullet\bullet\bullet\bullet$ & $\bullet\bullet\bullet$ \\
Riesgo de conflictos & $\bullet\bullet$ & $\bullet\bullet\bullet\bullet$ \\
Adaptación preferencias & $\bullet$ & $\bullet\bullet\bullet\bullet\bullet$ \\
Transparencia & $\bullet\bullet\bullet\bullet$ & $\bullet\bullet\bullet\bullet\bullet$ \\
Complejidad logística & $\bullet\bullet\bullet\bullet\bullet$ & $\bullet\bullet\bullet$ \\
\midrule
\textbf{Recomendación} & Aceptable & \textbf{Preferible} \\
\bottomrule
\end{tabularx}
\caption{Comparación de enfoques}
\end{table}

\subsection{Cuándo Usar Cada Método}

\textbf{Asignación Automática es mejor si}:
\begin{itemize}
    \item Los copropietarios viven en ciudades/países diferentes
    \item No hay posibilidad de reunirse presencial o virtualmente
    \item Se busca velocidad sobre consenso
    \item Los propietarios son inversionistas puros (no usarán la propiedad)
\end{itemize}

\textbf{Selección Voluntaria es mejor si}:
\begin{itemize}
    \item Los copropietarios planean usar la propiedad regularmente
    \item Hay voluntad de coordinarse para el Draft
    \item Se valora la satisfacción y aceptación sobre rapidez
    \item Existen preferencias específicas (calendario escolar, vacaciones laborales, etc.)
\end{itemize}

% ============================================================================
% CHECKLIST DE IMPLEMENTACIÓN
% ============================================================================
\section{Checklist de Implementación}

\subsection{30 Días Antes del Draft}

\begin{itemize}
    \item[$\square$] Clasificar las 52 semanas por temporada (ALTA/MEDIA/BAJA)
    \item[$\square$] Publicar calendario de semanas disponibles
    \item[$\square$] Determinar orden inicial (A, B, C, D) mediante sorteo
    \item[$\square$] Agendar fecha y hora del Draft
    \item[$\square$] Confirmar asistencia de los 4 copropietarios
    \item[$\square$] Contratar notario o designar facilitador neutral
    \item[$\square$] Preparar plataforma virtual (si es remoto)
\end{itemize}

\subsection{7 Días Antes del Draft}

\begin{itemize}
    \item[$\square$] Enviar recordatorio con fecha, hora y plataforma
    \item[$\square$] Compartir reglas del procedimiento
    \item[$\square$] Solicitar poderes notariales de ausentes
    \item[$\square$] Preparar hoja de registro para documentar selecciones
    \item[$\square$] Configurar grabación de audio/video
\end{itemize}

\subsection{Día del Draft}

\begin{itemize}
    \item[$\square$] Verificar asistencia de los 4 copropietarios
    \item[$\square$] Iniciar grabación oficial
    \item[$\square$] Leer reglas y orden serpiente
    \item[$\square$] Ejecutar las 13 rondas de selección
    \item[$\square$] Confirmar calendario final con todos
    \item[$\square$] Firmar acta de conformidad
\end{itemize}

\subsection{48 Horas Después del Draft}

\begin{itemize}
    \item[$\square$] Enviar calendario final a todos los copropietarios
    \item[$\square$] Publicar en plataforma del fideicomiso
    \item[$\square$] Notificar a empresa gestora
    \item[$\square$] Archivar grabación y acta notarial
    \item[$\square$] Actualizar sistema de reservaciones
\end{itemize}

% ============================================================================
% CONCLUSIÓN
% ============================================================================
\section{Conclusión}

El procedimiento de selección voluntaria con orden serpiente representa el \textbf{equilibrio perfecto} entre equidad matemática y satisfacción humana. Aunque requiere mayor coordinación logística que una asignación automática, los beneficios en términos de aceptación y transparencia justifican ampliamente el esfuerzo.

\subsection{Recomendación Final}

Para proyectos de propiedad fraccional donde los copropietarios planean usar la propiedad regularmente, el \textbf{Draft de Selección Voluntaria} es la opción superior. Genera:

\begin{itemize}
    \item Mayor satisfacción al respetar preferencias individuales
    \item Menor conflictividad al ser un proceso transparente
    \item Mejor relación a largo plazo entre copropietarios
    \item Sensación de control sobre su inversión
\end{itemize}

\vspace{1cm}

\begin{cajadestacado}{Para el Fideicomiso}
Se recomienda incorporar este procedimiento en el \textbf{Reglamento Interno del Fideicomiso} como la metodología oficial de asignación anual de semanas, estableciendo:

\begin{itemize}
    \item Fecha fija: 1er sábado de noviembre de cada año
    \item Facilitador oficial: Representante del banco fiduciario
    \item Rotación del orden inicial cada año
    \item Grabación obligatoria del proceso
    \item Acta de conformidad firmada por los 4 copropietarios
\end{itemize}
\end{cajadestacado}

\vfill

\begin{center}
\textcolor{verdeoscuro}{\rule{0.8\textwidth}{0.5pt}}\\[0.5cm]
\textbf{\large ANEXO TÉCNICO}\\
\textit{Documento complementario al Reporte Principal}\\
Propiedad Fraccional Mérida, Yucatán\\[0.3cm]
{\small \today}
\end{center}

\end{document}
