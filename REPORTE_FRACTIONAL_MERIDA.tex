% ============================================================================
% PROYECTO DE COMERCIALIZACIÓN - PROPIEDAD FRACCIONAL
% Departamento en Mérida, Yucatán
% ============================================================================
\documentclass[11pt,letterpaper,oneside]{scrartcl}

% ============================================================================
% PAQUETES ESENCIALES
% ============================================================================
\usepackage[utf8]{inputenc}
\usepackage[spanish,mexico]{babel}
\usepackage[T1]{fontenc}
\usepackage{lmodern}

% Matemáticas y símbolos
\usepackage{amsmath,amssymb}
\usepackage{siunitx}
\sisetup{
    group-separator={,},
    group-minimum-digits=4,
    output-decimal-marker={.}
}

% Geometría y márgenes
\usepackage[letterpaper,margin=2.5cm,top=2cm,bottom=2.5cm]{geometry}

% Gráficos y color
\usepackage{graphicx}
\usepackage{xcolor}
\definecolor{turquesa}{RGB}{0,168,150}      % Verde turquesa Yucatán
\definecolor{verdeoscuro}{RGB}{0,100,80}    % Verde oscuro profesional
\definecolor{gris}{RGB}{64,64,64}           % Gris texto
\definecolor{grisclaro}{RGB}{245,245,245}   % Gris fondo
\definecolor{naranja}{RGB}{255,140,0}       % Naranja para alertas

% Tablas profesionales
\usepackage{booktabs}
\usepackage{tabularx}
\usepackage{multirow}
\usepackage{array}

% Cajas y destacados
\usepackage[most]{tcolorbox}
\tcbuselibrary{skins}

% Enlaces e hipervínculos
\usepackage[hidelinks,colorlinks=true,linkcolor=verdeoscuro,urlcolor=turquesa,citecolor=turquesa]{hyperref}

% Bibliografía moderna con biblatex
\usepackage[style=numeric,sorting=none,backend=biber]{biblatex}
\addbibresource{referencias_fractional.bib}

% Encabezados y pies de página
\usepackage{scrlayer-scrpage}
\clearpairofpagestyles
\ihead{\color{gris}\small Propiedad Fraccional Mérida}
\ohead{\color{gris}\small \today}
\cfoot{\pagemark}
\setkomafont{pagehead}{\small\normalfont}
\setkomafont{pagefoot}{\small\normalfont}

% Formato de secciones
\addtokomafont{section}{\color{verdeoscuro}}
\addtokomafont{subsection}{\color{turquesa}}
\RedeclareSectionCommand[
  beforeskip=1.5\baselineskip,
  afterskip=0.5\baselineskip]{section}
\RedeclareSectionCommand[
  beforeskip=1\baselineskip,
  afterskip=0.3\baselineskip]{subsection}

% Listas personalizadas
\usepackage{enumitem}
\setlist[itemize]{leftmargin=*,labelsep=0.5em,itemsep=0.3em}
\setlist[enumerate]{leftmargin=*,labelsep=0.5em,itemsep=0.3em}

% Formato de moneda
\newcommand{\MXN}[1]{\$\num{#1}~MXN}
\newcommand{\USD}[1]{\$\num{#1}~USD}

% ============================================================================
% CAJAS PERSONALIZADAS
% ============================================================================
\newtcolorbox{cajaresumen}{
    colback=grisclaro,
    colframe=turquesa,
    fonttitle=\bfseries\color{white},
    coltitle=white,
    colbacktitle=turquesa,
    title=Resumen Ejecutivo,
    enhanced,
    attach boxed title to top left={yshift=-2mm,xshift=5mm},
    boxed title style={sharp corners}
}

\newtcolorbox{cajadestacado}[1]{
    colback=grisclaro,
    colframe=verdeoscuro,
    fonttitle=\bfseries\color{white},
    coltitle=white,
    colbacktitle=verdeoscuro,
    title=#1,
    enhanced,
    attach boxed title to top left={yshift=-2mm,xshift=5mm},
    boxed title style={sharp corners}
}

\newtcolorbox{cajaalerta}{
    colback=white,
    colframe=naranja,
    fonttitle=\bfseries\color{white},
    coltitle=white,
    colbacktitle=naranja,
    title=Punto Clave,
    enhanced,
    attach boxed title to top left={yshift=-2mm,xshift=5mm},
    boxed title style={sharp corners}
}

% ============================================================================
% DOCUMENTO
% ============================================================================
\begin{document}

% ----------------------------------------------------------------------------
% PORTADA
% ----------------------------------------------------------------------------
\begin{titlepage}
    \centering
    \vspace*{2cm}
    
    {\Huge\bfseries\color{verdeoscuro} PROYECTO DE\\[0.3cm] COMERCIALIZACIÓN}
    
    \vspace{0.8cm}
    
    {\LARGE\color{turquesa} Propiedad Fraccional\\[0.3cm]}
    {\Large Mérida, Yucatán}
    
    \vspace{1.2cm}
    
    {\large\color{gris} Análisis de Modelo de Negocio\\[0.2cm]
    Sistema de Distribución Equitativa\\[0.2cm]
    Proyecciones Financieras}
    
    \vspace{1.5cm}
    
    \begin{tcolorbox}[colback=grisclaro,colframe=turquesa,width=0.75\textwidth,arc=0mm]
        \centering
        \textbf{\large Información Clave}\\[0.4cm]
        \begin{tabular}{rl}
            \textbf{Inversión por Ticket:} & \MXN{900000} \\[0.15cm]
            \textbf{Número de Tickets:} & 4 copropietarios \\[0.15cm]
            \textbf{Derechos de Uso:} & 13 semanas/año por ticket \\[0.15cm]
            \textbf{ROI Proyectado:} & 8.94\% - 12.05\% anual \\
        \end{tabular}
    \end{tcolorbox}
    
    \vspace{1.2cm}
    
    {\large\color{gris} \today}
    
    \vspace{1cm}
    
    \begin{center}
        \color{gris}
        \textit{Documento preparado para}\\[0.2cm]
        {\Large\bfseries\color{verdeoscuro} Ing. Daniel Castillo Villalón}\\[0.5cm]
        
        \textit{por}\\[0.2cm]
        {\large\bfseries Sergio Muñoz de Alba Medrano}\\[0.15cm]
        Consultor Independiente\\[0.3cm]
        
        {\small
        +52 999 200 5550 $\cdot$ smunozam@gmail.com
        }
    \end{center}
    
\end{titlepage}

% ----------------------------------------------------------------------------
% TABLA DE CONTENIDOS
% ----------------------------------------------------------------------------
\tableofcontents
\newpage

% ============================================================================
% RESUMEN EJECUTIVO
% ============================================================================
\section{Resumen Ejecutivo}

\begin{cajaresumen}
Este documento presenta el análisis integral del proyecto de comercialización de una propiedad en modalidad fraccional (co-ownership) ubicada en la zona residencial norte de Mérida, Yucatán. El modelo ofrece una solución innovadora que combina:

\begin{itemize}
    \item \textbf{Inversión accesible}: 25\% del costo de propiedad completa
    \item \textbf{Flexibilidad dual}: Opción de uso personal o generación de rendimientos
    \item \textbf{Sistema equitativo}: Distribución justa de semanas mediante rotación serpiente
    \item \textbf{Rendimientos competitivos}: 8.94\% - 12.05\% anual proyectado
\end{itemize}
\end{cajaresumen}

\subsection{Características de la Propiedad}

\begin{itemize}
    \item \textbf{Ubicación}: Zona residencial norte de Mérida, Yucatán
    \item \textbf{Superficie}: 100 m² de construcción
    \item \textbf{Capacidad}: 4 personas (pax)
    \item \textbf{Estado}: En desarrollo
\end{itemize}

\subsection{Estructura de Inversión}

\begin{table}[h]
\centering
\begin{tabular}{lr}
\toprule
\textbf{Concepto} & \textbf{Monto} \\
\midrule
Precio por Ticket & \MXN{900000} \\
Número de Tickets & 4 \\
Capital Total Recaudado & \MXN{3600000} \\
Derechos de Uso & 13 semanas/ticket/año \\
\bottomrule
\end{tabular}
\caption{Estructura básica de inversión}
\end{table}

\subsection{Propuesta de Valor para Copropietarios}

El modelo ofrece tres opciones de aprovechamiento según las necesidades de cada inversionista:

\begin{enumerate}
    \item \textbf{Solo Inversión} (0 semanas de uso):
    \begin{itemize}
        \item Rendimiento: 8.94\% - 12.05\% anual
        \item Ingreso pasivo: \MXN{80464} - \MXN{108410} por año
        \item Cero responsabilidad operativa
    \end{itemize}
    
    \item \textbf{Uso Mixto} (ejemplo: 4 semanas uso + 9 semanas renta):
    \begin{itemize}
        \item Disfrute: 28 días/año de vacaciones en propiedad propia
        \item Rendimiento: $\sim$\MXN{60000} - \MXN{75000} por año
        \item Ahorro: $\sim$\MXN{50000} en hospedaje vs. hoteles
    \end{itemize}
    
    \item \textbf{Solo Uso Personal} (13 semanas):
    \begin{itemize}
        \item Disfrute: 91 días/año de vacaciones
        \item Valor de uso: $\sim$\MXN{180000} - \MXN{230000} en hospedaje equivalente
        \item ROI indirecto: 20-25\% en valor de uso vs. hoteles
    \end{itemize}
\end{enumerate}

% ============================================================================
% SISTEMA DE DISTRIBUCIÓN SERPIENTE
% ============================================================================
\section{Sistema de Rotación Serpiente}

\subsection{El Desafío Fundamental}

\begin{cajaalerta}
\textbf{¿Cómo dividir las 52 semanas de manera ``salomónica'' entre 4 tickets para que todos tengan acceso equitativo a buenas temporadas?}

Esta es la pregunta central que determina el éxito del modelo de copropietarios.
\end{cajaalerta}

No todas las semanas del año tienen el mismo valor. En Mérida, existen tres temporadas claramente diferenciadas:

\begin{itemize}
    \item \textbf{Temporada ALTA (26 semanas)}: Diciembre-Marzo, Julio-Agosto
    \begin{itemize}
        \item Clima ideal (18-28°C), menor probabilidad de lluvias
        \item Mayor demanda turística (Navidad, Año Nuevo, Semana Santa)
        \item Mejores tarifas de renta: \MXN{2500}/noche
    \end{itemize}
    
    \item \textbf{Temporada MEDIA (14 semanas)}: Abril, Mayo parcial, Octubre-Noviembre
    \begin{itemize}
        \item Clima agradable en transición
        \item Demanda moderada, eventos culturales
        \item Tarifas intermedias: \MXN{1800}/noche
    \end{itemize}
    
    \item \textbf{Temporada BAJA (12 semanas)}: Mayo tardío, Junio, Septiembre
    \begin{itemize}
        \item Temporada de lluvias (temperaturas $>$35°C)
        \item Menor demanda, pero paisajes exuberantes
        \item Tarifas bajas: \MXN{1200}/noche
    \end{itemize}
\end{itemize}

\subsection{Solución Adoptada: Sistema Serpiente (Zig-Zag)}

\begin{cajadestacado}{¿Cómo Funciona el Sistema Serpiente?}
El sistema invierte el orden de asignación en cada ronda, garantizando que ningún propietario tenga ventaja permanente:

\vspace{0.5cm}
\textbf{Ronda 1}: A $\rightarrow$ B $\rightarrow$ C $\rightarrow$ D (orden normal)\\
\textbf{Ronda 2}: D $\rightarrow$ C $\rightarrow$ B $\rightarrow$ A (orden \textit{invertido})\\
\textbf{Ronda 3}: A $\rightarrow$ B $\rightarrow$ C $\rightarrow$ D (orden normal)\\
\textbf{Ronda 4}: D $\rightarrow$ C $\rightarrow$ B $\rightarrow$ A (orden \textit{invertido})

\vspace{0.3cm}
Este patrón continúa a lo largo de las 52 semanas del año.
\end{cajadestacado}

\subsubsection{Distribución Resultante}

Al aplicar el sistema serpiente, la distribución queda perfectamente balanceada:

\begin{table}[h]
\centering
\begin{tabular}{lccc}
\toprule
\textbf{Ticket} & \textbf{Temp. ALTA} & \textbf{Temp. MEDIA} & \textbf{Temp. BAJA} \\
\midrule
Ticket A & 7 semanas & 3 semanas & 3 semanas \\
Ticket B & 7 semanas & 3 semanas & 3 semanas \\
Ticket C & 6 semanas & 4 semanas & 3 semanas \\
Ticket D & 6 semanas & 4 semanas & 3 semanas \\
\midrule
\textbf{TOTAL} & \textbf{26 semanas} & \textbf{14 semanas} & \textbf{12 semanas} \\
\bottomrule
\end{tabular}
\caption{Distribución de semanas por temporada y ticket}
\end{table}

\vspace{0.5cm}
\textbf{Nota sobre equidad}: La diferencia de 1 semana en temporada ALTA es matemáticamente inevitable cuando se dividen 26 semanas entre 4 propietarios (26 $\div$ 4 = 6.5). Sin embargo, el sistema serpiente compensa esta pequeña variación otorgando 1 semana adicional de temporada MEDIA a los Tickets C y D.

\subsection{Ventajas del Sistema Serpiente}

\begin{enumerate}
    \item \textbf{Equidad Matemática Perfecta}
    \begin{itemize}
        \item Ningún propietario tiene ventaja permanente en la posición de selección
        \item Todos eligen primero en algunas rondas y último en otras
        \item Distribución objetiva y demostrable
    \end{itemize}
    
    \item \textbf{Transparencia Total}
    \begin{itemize}
        \item El patrón es predecible y fácil de auditar
        \item Cualquier copropietario puede verificar la distribución en Excel
        \item No hay discrecionalidad ni favoritismos posibles
    \end{itemize}
    
    \item \textbf{Minimización de Conflictos}
    \begin{itemize}
        \item Al ser matemáticamente justo, reduce quejas entre socios
        \item Elimina la percepción de que algunos ``siempre ganan''
        \item Genera confianza a largo plazo
    \end{itemize}
    
    \item \textbf{Distribución Temporal Equilibrada}
    \begin{itemize}
        \item Todos tienen semanas distribuidas a lo largo del año
        \item No se concentran en un solo bloque (mayor flexibilidad)
        \item Permite planear vacaciones según agenda personal
    \end{itemize}
    
    \item \textbf{Rotación Anual para Equidad Multi-Año}
    \begin{itemize}
        \item Año 1: Orden inicial A$\rightarrow$B$\rightarrow$C$\rightarrow$D
        \item Año 2: Orden inicial B$\rightarrow$C$\rightarrow$D$\rightarrow$A
        \item Año 3: Orden inicial C$\rightarrow$D$\rightarrow$A$\rightarrow$B
        \item Año 4: Orden inicial D$\rightarrow$A$\rightarrow$B$\rightarrow$C
        \item En 4 años, todos han sido A, B, C y D
    \end{itemize}
\end{enumerate}

\subsection{Comparación con Sistemas Alternativos}

\begin{table}[h]
\centering
\small
\begin{tabularx}{\textwidth}{Xccc}
\toprule
\textbf{Criterio} & \textbf{Rotación Simple} & \textbf{Sorteo Aleatorio} & \textbf{Serpiente (Zig-Zag)} \\
\midrule
Equidad en temporada alta & Buena & Variable (suerte) & Excelente \\
Predictibilidad & Alta & Baja & Alta \\
Percepción de justicia & Media & Baja & Muy alta \\
Conflictos potenciales & Medios-Altos & Altos & Mínimos \\
Facilidad implementación & Muy fácil & Compleja & Fácil \\
Auditable/Verificable & Sí & Sí (requiere acta) & Sí (Excel) \\
\midrule
\textbf{Recomendación} & Buena & Regular & \textbf{Excelente} \\
\bottomrule
\end{tabularx}
\caption{Comparación de sistemas de distribución de semanas}
\end{table}

\subsubsection{Problema del Sistema de Rotación Simple}

El sistema simple (A-B-C-D-A-B-C-D...) tiene un defecto crítico:

\begin{itemize}
    \item Ticket A \textbf{siempre} elige primero en su turno
    \item Ticket D \textbf{siempre} elige último en su turno
    \item Si Navidad/Año Nuevo caen en ``semanas A'', Ticket A las monopoliza \textit{cada año}
    \item Genera resentimiento y percepción de inequidad permanente
\end{itemize}

\subsubsection{Problema del Sorteo Aleatorio}

Aunque parece ``justo'' por ser aleatorio, tiene desventajas severas:

\begin{itemize}
    \item Resultados impredecibles año con año
    \item Un propietario puede tener ``mala suerte'' varios años consecutivos
    \item Requiere proceso de sorteo formal cada año (costos, logística)
    \item Genera ansiedad e incertidumbre en planificación familiar
    \item No aprovecha la predictibilidad que valoran los inversionistas
\end{itemize}

% ============================================================================
% ANÁLISIS FINANCIERO
% ============================================================================
\section{Análisis Financiero y Proyecciones}

\subsection{Parámetros Base del Modelo}

\begin{table}[h]
\centering
\begin{tabular}{lr}
\toprule
\textbf{Parámetro} & \textbf{Valor} \\
\midrule
Inversión por Ticket & \MXN{900000} \\
Número de Tickets & 4 \\
Capital Total Recaudado & \MXN{3600000} \\
Derechos de Uso & 13 semanas/ticket/año \\
\midrule
Tarifa Temporada ALTA & \MXN{2500}/noche \\
Tarifa Temporada MEDIA & \MXN{1800}/noche \\
Tarifa Temporada BAJA & \MXN{1200}/noche \\
\midrule
Gastos Operativos & 28\% de ingresos brutos \\
\bottomrule
\end{tabular}
\caption{Parámetros financieros base del modelo}
\end{table}

\subsection{Tres Escenarios de Proyección}

Se presentan tres escenarios con diferentes niveles de ocupación para evaluar el rango de rendimientos esperados:

\begin{table}[h]
\centering
\small
\begin{tabularx}{\textwidth}{Xcccccc}
\toprule
\textbf{Escenario} & \textbf{Ocup. ALTA} & \textbf{Ocup. MEDIA} & \textbf{Ocup. BAJA} & \textbf{Ing. Brutos} & \textbf{Ing. Netos} & \textbf{ROI} \\
\midrule
Conservador & 70\% & 50\% & 40\% & \MXN{447020} & \MXN{321854} & \textbf{8.94\%} \\
Moderado & 80\% & 65\% & 50\% & \MXN{529060} & \MXN{380923} & \textbf{10.58\%} \\
Optimista & 90\% & 75\% & 60\% & \MXN{602280} & \MXN{433642} & \textbf{12.05\%} \\
\bottomrule
\end{tabularx}
\caption{Resumen de los tres escenarios financieros}
\end{table}

\subsection{Escenario 1: Conservador}

\textbf{Supuestos de ocupación}: 70\% temporada alta, 50\% media, 40\% baja

\subsubsection{Flujo de Ingresos}

\begin{itemize}
    \item \textbf{Temporada ALTA} (26 semanas $\times$ 7 días = 182 noches):
    \begin{itemize}
        \item 70\% ocupación $\times$ 182 noches = 127.4 noches rentadas
        \item 127.4 noches $\times$ \MXN{2500}/noche = \textbf{\MXN{318500}}
    \end{itemize}
    
    \item \textbf{Temporada MEDIA} (14 semanas $\times$ 7 días = 98 noches):
    \begin{itemize}
        \item 50\% ocupación $\times$ 98 noches = 49 noches rentadas
        \item 49 noches $\times$ \MXN{1800}/noche = \textbf{\MXN{88200}}
    \end{itemize}
    
    \item \textbf{Temporada BAJA} (12 semanas $\times$ 7 días = 84 noches):
    \begin{itemize}
        \item 40\% ocupación $\times$ 84 noches = 33.6 noches rentadas
        \item 33.6 noches $\times$ \MXN{1200}/noche = \textbf{\MXN{40320}}
    \end{itemize}
\end{itemize}

\textbf{Ingresos Brutos Totales}: \MXN{447020}

\subsubsection{Gastos Operativos (28\%)}

\begin{table}[h]
\centering
\begin{tabular}{lr}
\toprule
\textbf{Concepto} & \textbf{Monto} \\
\midrule
Empresa Gestora (15\% de ingresos) & \MXN{67053} \\
Mantenimiento \& Reparaciones (5\%) & \MXN{22351} \\
Servicios Públicos (4\%) & \MXN{17881} \\
Seguros \& Fideicomiso (4\%) & \MXN{17881} \\
\midrule
\textbf{Total Gastos Operativos} & \MXN{125166} \\
\bottomrule
\end{tabular}
\caption{Desglose de gastos operativos - Escenario Conservador}
\end{table}

\subsubsection{Rendimiento Anual}

\begin{align*}
\text{Ingresos Netos} &= \text{\MXN{447020}} - \text{\MXN{125166}} = \textbf{\MXN{321854}} \\
\text{Por Ticket} &= \text{\MXN{321854}} \div 4 = \textbf{\MXN{80464}} \\
\text{Tasa de Rendimiento} &= \left(\frac{\text{\MXN{80464}}}{\text{\MXN{900000}}}\right) \times 100 = \textbf{8.94\% anual}
\end{align*}

\textbf{Interpretación}: Este escenario asume ocupación cautelosa, especialmente en temporadas bajas. Es realista para el año 1 de operación mientras se consolida la reputación en plataformas. Aun así, ofrece rendimientos competitivos.

\subsection{Escenario 2: Moderado (Recomendado)}

\textbf{Supuestos de ocupación}: 80\% temporada alta, 65\% media, 50\% baja

\subsubsection{Flujo de Ingresos}

\begin{itemize}
    \item \textbf{Temporada ALTA}: 80\% $\times$ 182 noches = 145.6 noches $\rightarrow$ \textbf{\MXN{364000}}
    \item \textbf{Temporada MEDIA}: 65\% $\times$ 98 noches = 63.7 noches $\rightarrow$ \textbf{\MXN{114660}}
    \item \textbf{Temporada BAJA}: 50\% $\times$ 84 noches = 42 noches $\rightarrow$ \textbf{\MXN{50400}}
\end{itemize}

\textbf{Ingresos Brutos Totales}: \MXN{529060}

\subsubsection{Rendimiento Anual}

\begin{align*}
\text{Ingresos Netos} &= \text{\MXN{529060}} - \text{\MXN{148137}} = \textbf{\MXN{380923}} \\
\text{Por Ticket} &= \text{\MXN{380923}} \div 4 = \textbf{\MXN{95231}} \\
\text{Tasa de Rendimiento} &= \left(\frac{\text{\MXN{95231}}}{\text{\MXN{900000}}}\right) \times 100 = \textbf{10.58\% anual}
\end{align*}

\begin{cajadestacado}{Escenario Recomendado para Presentar a Inversionistas}
Este escenario es REALISTA basado en:
\begin{itemize}
    \item Benchmarks de propiedades comparables en Airbnb Mérida
    \item Ocupación típica de propiedades de 4 pax bien gestionadas
    \item Crecimiento natural años 1-2 mientras la marca se consolida
    \item \textbf{10.58\% anual supera CETES (10\%) con activo tangible}
\end{itemize}
\end{cajadestacado}

\subsection{Escenario 3: Optimista}

\textbf{Supuestos de ocupación}: 90\% temporada alta, 75\% media, 60\% baja

\subsubsection{Flujo de Ingresos}

\begin{itemize}
    \item \textbf{Temporada ALTA}: 90\% $\times$ 182 noches = 163.8 noches $\rightarrow$ \textbf{\MXN{409500}}
    \item \textbf{Temporada MEDIA}: 75\% $\times$ 98 noches = 73.5 noches $\rightarrow$ \textbf{\MXN{132300}}
    \item \textbf{Temporada BAJA}: 60\% $\times$ 84 noches = 50.4 noches $\rightarrow$ \textbf{\MXN{60480}}
\end{itemize}

\textbf{Ingresos Brutos Totales}: \MXN{602280}

\subsubsection{Rendimiento Anual}

\begin{align*}
\text{Ingresos Netos} &= \text{\MXN{602280}} - \text{\MXN{168638}} = \textbf{\MXN{433642}} \\
\text{Por Ticket} &= \text{\MXN{433642}} \div 4 = \textbf{\MXN{108410}} \\
\text{Tasa de Rendimiento} &= \left(\frac{\text{\MXN{108410}}}{\text{\MXN{900000}}}\right) \times 100 = \textbf{12.05\% anual}
\end{align*}

\textbf{Interpretación}: Escenario ASPIRACIONAL que requiere gestión proactiva, marketing fuerte, reseñas 5 estrellas consistentes y estrategia de pricing dinámico optimizada. Posible en años 3+ de operación. Comparable a retornos de REITs de lujo.

\subsection{Comparativa con Alternativas de Inversión}

\begin{table}[h]
\centering
\begin{tabularx}{\textwidth}{Xccc}
\toprule
\textbf{Instrumento} & \textbf{Retorno Anual} & \textbf{Ventajas} & \textbf{Desventajas} \\
\midrule
CETES 2025 & $\sim$10\% & Seguridad, liquidez & Sin activo tangible \\
Fondos de Inversión & 8-10\% & Diversificación & Comisiones altas \\
Bonos Corporativos & 8-12\% & Fijo, predecible & Riesgo incumplimiento \\
\midrule
\textbf{Propiedad Fraccional} & \textbf{8.94\%-12.05\%} & \textbf{Activo tangible + uso} & \textbf{Requiere gestión} \\
\bottomrule
\end{tabularx}
\caption{Comparación con alternativas de inversión en México}
\end{table}

\subsection{Análisis de Sensibilidad}

\subsubsection{Escenario: Baja ocupación en temporada alta (60\% vs. 80\% esperado)}

\begin{itemize}
    \item Impacto: Caída de $\sim$15\% en ingresos anuales
    \item Rendimiento nuevo: $\sim$9\% anual (aún superior a CETES)
    \item Conclusión: El modelo es resiliente ante ocupación menor a la esperada
\end{itemize}

\subsubsection{Escenario: Aumento de gastos a 35\% (por inflación/reparaciones)}

\begin{itemize}
    \item Impacto: Gastos suben de 28\% a 35\% de ingresos
    \item Rendimiento nuevo: $\sim$8.5\% anual (escenario moderado)
    \item Conclusión: Aun con costos elevados, rendimientos competitivos
\end{itemize}

\subsubsection{Escenario: Apreciación de propiedad 3\% anual}

\begin{itemize}
    \item Plusvalía anual: \MXN{108000}
    \item ROI TOTAL: 10.58\% (rendimiento) + 3\% (apreciación) = \textbf{13.58\% anual}
    \item Conclusión: La plusvalía inmobiliaria amplifica significativamente el retorno total
\end{itemize}

% ============================================================================
% ESTRUCTURA LEGAL
% ============================================================================
\section{Estructura Legal y Administrativa}

\subsection{Fideicomiso Inmobiliario (Estructura Recomendada)}

\begin{cajadestacado}{¿Por Qué Fideicomiso en Lugar de Tiempo Compartido Tradicional?}
El fideicomiso inmobiliario ofrece ventajas legales y patrimoniales superiores:

\begin{itemize}
    \item Otorga \textbf{derechos reales de propiedad}, no solo derechos de uso temporal
    \item El banco fiduciario mantiene el título formal y emite 4 derechos fiduciarios iguales
    \item Cada derecho es vendible, heredable y aprecia con el valor de la propiedad
    \item Se diferencia del ``tiempo compartido tradicional'' (mejor protección legal)
    \item Requerido para inversionistas extranjeros en zonas restringidas
\end{itemize}
\end{cajadestacado}

\subsection{Derechos Incluidos en Cada Derecho Fiduciario}

Cada uno de los 4 derechos fiduciarios incluye:

\begin{enumerate}
    \item \textbf{Participación Patrimonial}: 25\% del valor de la propiedad
    \item \textbf{Derecho de Uso}: 13 semanas específicas anuales según calendario serpiente
    \item \textbf{Derecho a Rendimientos}: Participación proporcional en ingresos netos de rentas
    \item \textbf{Derecho de Venta}: Posibilidad de transferir el derecho fiduciario
    \item \textbf{Derecho de Preferencia}: Prioridad de compra cuando otro copropietario vende
    \item \textbf{Derecho de Herencia}: El derecho fiduciario es transmisible a herederos
    \item \textbf{Voto en Decisiones}: Participación equitativa en juntas de copropietarios
\end{enumerate}

\subsection{Reglamento Interno del Fideicomiso}

El fideicomiso debe incluir un reglamento interno detallado que establezca:

\begin{itemize}
    \item Calendario de asignación de semanas y sistema de rotación serpiente
    \item Procedimientos para reservación y uso (deadline 60 días)
    \item Cuotas de mantenimiento y su distribución entre copropietarios
    \item Mecanismos de resolución de conflictos (arbitraje)
    \item Procedimientos para modificaciones mayores a la propiedad
    \item Requisitos de quórum para decisiones importantes (mínimo 3 de 4 votos)
    \item Protocolo de venta/transferencia de derechos fiduciarios
    \item Obligaciones de la empresa gestora y KPIs de desempeño
    \item Fondo de reserva para contingencias (6 meses de gastos operativos)
\end{itemize}

\subsection{Costos del Fideicomiso}

\subsubsection{Costos Iniciales (Una Sola Vez)}

\begin{table}[h]
\centering
\begin{tabular}{lr}
\toprule
\textbf{Concepto} & \textbf{Monto} \\
\midrule
Honorarios de constitución & \MXN{45000} - \MXN{50000} \\
Gastos notariales & \MXN{15000} - \MXN{18000} \\
Asesoría legal y financiera & \MXN{20000} - \MXN{25000} \\
\midrule
\textbf{Total Costos Iniciales} & \textbf{\MXN{80000} - \MXN{93000}} \\
\midrule
Por Ticket (dividido entre 4) & \MXN{20000} - \MXN{23250} \\
Como \% de inversión & 2.2\% - 2.6\% \\
\bottomrule
\end{tabular}
\caption{Costos iniciales del fideicomiso}
\end{table}

\subsubsection{Costos Recurrentes Anuales}

\begin{table}[h]
\centering
\begin{tabular}{lc}
\toprule
\textbf{Escenario} & \textbf{Costo Anual Total} \\
\midrule
Conservador (1.5\% sobre patrimonio \MXN{3600000}) & \MXN{54000} \\
Moderado (2.0\% sobre patrimonio) & \MXN{72000} \\
Alto (2.5\% sobre patrimonio) & \MXN{90000} \\
\midrule
\textbf{Por Ticket (dividido entre 4)} & \\
Conservador & \MXN{13500}/año \\
Moderado & \MXN{18000}/año \\
Alto & \MXN{22500}/año \\
\bottomrule
\end{tabular}
\caption{Costos recurrentes anuales del fideicomiso}
\end{table}

\textbf{Nota importante}: Estos costos están incluidos en el 28\% de gastos operativos de las proyecciones financieras, por lo que NO afectan adicionalmente el rendimiento calculado.

\subsection{Bancos Fiduciarios Recomendados para Mérida}

\subsubsection{Opción 1: BIM (Banco Inmobiliario Mexicano) - RECOMENDADA}

\textbf{Ventajas decisivas}:
\begin{itemize}
    \item Especialización exclusiva en fideicomisos inmobiliarios
    \item Tarifas más bajas del mercado (1.46\% - 5.25\%, tu proyecto: 1.5\% - 1.8\%)
    \item Experiencia comprobada en proyectos de desarrollo y copropiedad
    \item Entiende perfectamente el modelo de propiedad fraccional
\end{itemize}

\textbf{Estimado para tu proyecto}: \MXN{54000} - \MXN{65000} anuales

\subsubsection{Opción 2: Banorte - SEGUNDA OPCIÓN}

\textbf{Ventajas decisivas}:
\begin{itemize}
    \item Presencia sólida en Mérida (múltiples sucursales)
    \item Anualidad fija sin incrementos (mayor certidumbre)
    \item Solidez institucional (uno de los mayores bancos de México)
    \item Integración de servicios complementarios
\end{itemize}

\textbf{Nota}: Las tarifas USD (\$464/año) aplican para zona restringida (extranjeros). Para fideicomiso de copropiedad nacional, cotizar directamente.

% ============================================================================
% REGLAS OPERATIVAS
% ============================================================================
\section{Reglas Operativas y Reservaciones}

\subsection{Proceso de Reservación Anual}

\begin{enumerate}
    \item \textbf{Publicación del calendario}: 1 de noviembre para el año siguiente
    \item \textbf{Notificación de intención de uso}: Los copropietarios deben confirmar con \textbf{60 días de anticipación}
    \item \textbf{Deadline general}: 1 de noviembre del año previo para planificación completa
    \item \textbf{Semanas no reclamadas}: Se rentan automáticamente a través de plataformas
    \item \textbf{Sin reclamaciones tardías}: No se aceptan solicitudes después del deadline
\end{enumerate}

\subsection{Intercambios Entre Propietarios}

Los intercambios de semanas están \textbf{permitidos} bajo estas condiciones:

\begin{itemize}
    \item Deben ser mutuamente acordados por escrito
    \item Notificación formal al fiduciario y empresa gestora
    \item Válidos únicamente para el año específico
    \item No afectan el calendario del año siguiente
\end{itemize}

\textbf{Ejemplo}: Ticket A tiene asignada la Semana 10 (marzo) pero prefiere la Semana 30 (julio). Ticket C tiene la Semana 30 pero necesita la Semana 10. Pueden intercambiar mediante acuerdo documentado.

\subsection{Distribución de Rendimientos}

\begin{itemize}
    \item \textbf{Cálculo}: Se realiza al cierre del año fiscal (31 de diciembre)
    \item \textbf{Deducción de gastos}: Se restan todos los costos operativos
    \item \textbf{Distribución proporcional}: Solo los tickets cuyas semanas fueron rentadas reciben ingresos
    \item \textbf{Pago}: Se efectúa en enero del año siguiente
    \item \textbf{Transparencia}: Estados financieros trimestrales para todos los copropietarios
\end{itemize}

\begin{cajaalerta}
\textbf{Regla Fundamental}: Si un copropietario usa una semana, \textbf{renuncia al ingreso} de esa semana específica. No se puede usar Y cobrar la misma semana.
\end{cajaalerta}

\subsection{Regla de la Semana 53 (Años Bisiestos)}

En años bisiestos (53 semanas), existen dos opciones:

\begin{enumerate}
    \item \textbf{Opción 1 - Pool común}: La semana 53 se renta automáticamente y el ingreso se divide equitativamente entre los 4 tickets
    \item \textbf{Opción 2 - Rotación}: Año 1 $\rightarrow$ Ticket A, Año 2 $\rightarrow$ Ticket B, Año 3 $\rightarrow$ Ticket C, Año 4 $\rightarrow$ Ticket D
\end{enumerate}

Se recomienda la Opción 1 para mantener total equidad.

% ============================================================================
% PLAN DE IMPLEMENTACIÓN
% ============================================================================
\section{Plan de Implementación}

\subsection{Fase 1: Investigación y Validación (4-6 semanas)}

\begin{itemize}
    \item Benchmark de tarifas de renta en propiedades comparables (Airbnb/Booking zona norte Mérida)
    \item Análisis de tasas de ocupación reales por temporada
    \item Cotización de al menos 3 empresas gestoras locales
    \item Estimación precisa de gastos operativos (predial, utilities, insurance)
    \item Validación de apetito de mercado (encuestas, focus groups con inversionistas potenciales)
\end{itemize}

\subsection{Fase 2: Estructura Legal y Financiera (6-8 semanas)}

\begin{itemize}
    \item Contratación de notario especializado en fideicomisos inmobiliarios
    \item Estructuración del fideicomiso con banco fiduciario (BIM o Banorte)
    \item Redacción detallada del reglamento interno
    \item Desarrollo de modelo financiero con análisis de sensibilidad
    \item Establecimiento del fondo de reserva (6 meses de gastos operativos)
\end{itemize}

\subsection{Fase 3: Materiales de Comercialización (4 semanas)}

\begin{itemize}
    \item Creación de pitch deck profesional (15-20 slides)
    \item Desarrollo de brochure digital interactivo con renders de la propiedad
    \item Producción de video tour virtual
    \item Diseño de calculadora de ROI personalizable para diferentes perfiles
    \item Preparación de documento FAQ (preguntas frecuentes)
    \item Borrador de contratos tipo con revisión legal
\end{itemize}

\subsection{Fase 4: Búsqueda de Inversionistas (8-12 semanas)}

\begin{itemize}
    \item Identificación y calificación de prospectos
    \item Presentaciones individuales personalizadas
    \item Facilitación del proceso de due diligence
    \item Negociación y cierre de las 4 ventas de tickets
    \item Ejecución del proceso de onboarding para copropietarios
\end{itemize}

\subsection{Fase 5: Implementación Operativa (continuo)}

\begin{itemize}
    \item Contratación de empresa gestora con track record comprobable
    \item Configuración de perfiles en plataformas (Airbnb, Booking, VRBO)
    \item Sesión fotográfica profesional de la propiedad
    \item Implementación de estrategia de pricing dinámico
    \item Establecimiento de sistema de monitoreo de KPIs
    \item Reuniones trimestrales de resultados con copropietarios
\end{itemize}

% ============================================================================
% PERFIL DEL INVERSIONISTA IDEAL
% ============================================================================
\section{Perfil del Inversionista Ideal}

\subsection{Características Demográficas}

\begin{itemize}
    \item \textbf{Edad}: 35-55 años
    \item \textbf{Ingreso familiar}: $>$\MXN{100000} mensuales
    \item \textbf{Ubicación}: Ciudad de México, Monterrey, Guadalajara, o extranjeros con conexión a Yucatán
\end{itemize}

\subsection{Perfil Conductual}

\begin{itemize}
    \item Ya viajan a Mérida 2 o más veces al año
    \item Buscan diversificación de portafolio más allá de instrumentos tradicionales
    \item Valoran experiencias de vacaciones familiares
    \item Tienen horizonte de inversión de mediano a largo plazo (5+ años)
    \item Aprecian transparencia en rendimientos y gestión profesional
\end{itemize}

\subsection{Características Psicográficas}

\begin{itemize}
    \item Prefieren activos tangibles sobre inversiones puramente financieras
    \item Valoran flexibilidad en decisiones (uso vs. renta según convenga cada año)
    \item Buscan comunidad de copropietarios con valores similares
    \item Aprecian gestión profesional que minimice su involucramiento operativo
\end{itemize}

% ============================================================================
% MITIGACIÓN DE RIESGOS
% ============================================================================
\section{Mitigación de Riesgos}

\subsection{Fondo de Reserva}

\begin{itemize}
    \item \textbf{Monto objetivo}: 6 meses de gastos operativos
    \item \textbf{Finalidad}: Cubrir períodos de baja ocupación, reparaciones inesperadas, transiciones de gestora
    \item \textbf{Capitalización}: 5\% de los ingresos netos anuales hasta alcanzar el monto objetivo
\end{itemize}

\subsection{Seguros Integrales}

\begin{enumerate}
    \item \textbf{Responsabilidad civil}: Cobertura por daños a terceros
    \item \textbf{Daños estructurales}: Incendio, inundación, sismos
    \item \textbf{Contenidos}: Equipamiento, mobiliario, electrodomésticos
\end{enumerate}

\subsection{Cláusulas de Protección}

\begin{itemize}
    \item \textbf{Derecho de preferencia}: Copropietarios tienen prioridad antes de venta externa
    \item \textbf{Resolución de disputas}: Proceso definido de arbitraje
    \item \textbf{Auditorías independientes}: Revisión financiera trimestral
    \item \textbf{Remoción de copropietario}: Proceso documentado para incumplimientos graves
\end{itemize}

\subsection{Diversificación de Ingresos}

\begin{itemize}
    \item Presencia en múltiples plataformas OTA (no dependencia de una sola)
    \item Relaciones con agencias de viajes locales
    \item Lista de clientes directos (bookings sin comisión)
    \item Oportunidades de partnerships corporativos
\end{itemize}

% ============================================================================
% CONCLUSIONES
% ============================================================================
\section{Conclusiones}

\subsection{Ventajas Competitivas del Modelo}

\begin{enumerate}
    \item \textbf{Accesibilidad Financiera}
    \begin{itemize}
        \item Inversión de \MXN{900000} vs. \MXN{3600000} de propiedad completa
        \item Democratiza acceso a bienes raíces en zona premium de Mérida
    \end{itemize}
    
    \item \textbf{Rendimientos Competitivos}
    \begin{itemize}
        \item 8.94\% - 12.05\% anual proyectado
        \item Supera instrumentos tradicionales con activo tangible
        \item Potencial de apreciación adicional del inmueble
    \end{itemize}
    
    \item \textbf{Flexibilidad Dual}
    \begin{itemize}
        \item Opción de elegir entre uso personal y generación de ingresos
        \item Adaptable a necesidades cambiantes de cada copropietario
    \end{itemize}
    
    \item \textbf{Ubicación Estratégica}
    \begin{itemize}
        \item Zona residencial norte de Mérida en crecimiento sostenido
        \item Alta demanda turística nacional e internacional
        \item Excelente conectividad y servicios
    \end{itemize}
    
    \item \textbf{Gestión Profesional}
    \begin{itemize}
        \item Empresa gestora libera a propietarios de responsabilidades operativas
        \item Transparencia total en finanzas y ocupación
        \item Monitoreo continuo de KPIs y optimización
    \end{itemize}
\end{enumerate}

\subsection{Sistema Serpiente: La Clave del Éxito}

El \textbf{Sistema de Rotación Serpiente} es el elemento diferenciador que garantiza la viabilidad a largo plazo del proyecto:

\begin{itemize}
    \item Equidad matemática perfecta en distribución de temporadas
    \item Elimina percepción de favoritismos entre copropietarios
    \item Genera confianza y reduce conflictos
    \item Fácil de implementar, auditar y verificar
    \item Utilizado exitosamente en desarrollos fraccionales en EE.UU. y Europa
\end{itemize}

\subsection{Recomendación Final}

El proyecto de propiedad fraccional en Mérida representa una \textbf{oportunidad viable y competitiva} en el mercado inmobiliario de Yucatán, diferenciándose sustancialmente del tiempo compartido tradicional al otorgar:

\begin{itemize}
    \item Derechos reales de propiedad (no solo derechos de uso)
    \item Participación en plusvalía del inmueble
    \item Opciones claras de exit strategy (venta, herencia)
    \item Transparencia total en operación y finanzas
\end{itemize}
REFERENCIAS BIBLIOGRÁFICAS
% ============================================================================
\newpage
\section{Referencias}

Las proyecciones financieras, análisis de mercado y marco legal presentados en este documento se fundamentan en las siguientes fuentes:

\printbibliography[heading=none]

% ============================================================================
% APÉNDICES
% ============================================================================
\newpage
\appendix
\section{Calendario Detallado de Semanas - Año 2026}

\begin{table}[h]
\centering
\tiny
\begin{tabular}{cllccc}
\toprule
\textbf{Sem.} & \textbf{Inicio} & \textbf{Fin} & \textbf{Mes} & \textbf{Temp.} & \textbf{Ticket} \\
\midrule
1 & 2026-01-05 & 2026-01-11 & Enero & ALTA & A \\
2 & 2026-01-12 & 2026-01-18 & Enero & ALTA & B \\
3 & 2026-01-19 & 2026-01-25 & Enero & ALTA & C \\
4 & 2026-01-26 & 2026-02-01 & Enero & ALTA & D \\
5 & 2026-02-02 & 2026-02-08 & Febrero & ALTA & D \\
6 & 2026-02-09 & 2026-02-15 & Febrero & ALTA & C \\
7 & 2026-02-16 & 2026-02-22 & Febrero & ALTA & B \\
8 & 2026-02-23 & 2026-03-01 & Febrero & ALTA & A \\
9 & 2026-03-02 & 2026-03-08 & Marzo & ALTA & A \\
10 & 2026-03-09 & 2026-03-15 & Marzo & ALTA & B \\
11 & 2026-03-16 & 2026-03-22 & Marzo & ALTA & C \\
12 & 2026-03-23 & 2026-03-29 & Marzo & ALTA & D \\
13 & 2026-03-30 & 2026-04-05 & Marzo & ALTA & D \\
14 & 2026-04-06 & 2026-04-12 & Abril & MEDIA & A \\
15 & 2026-04-13 & 2026-04-19 & Abril & MEDIA & B \\
16 & 2026-04-20 & 2026-04-26 & Abril & MEDIA & C \\
17 & 2026-04-27 & 2026-05-03 & Abril & MEDIA & D \\
\midrule
\multicolumn{6}{c}{\textit{... continuación para las 52 semanas completas}} \\
\bottomrule
\end{tabular}
\caption{Primeras 17 semanas del calendario 2026 (muestra representativa)}
\end{table}

\section{Contactos y Próximos Pasos}

\subsection{Instituciones Fiduciarias para Cotización}

\begin{enumerate}
    \item \textbf{BIM (Banco Inmobiliario Mexicano)}
    \begin{itemize}
        \item Contacto: [Pendiente obtener datos oficiales]
        \item Especificar: Fideicomiso copropiedad, 4 fideicomisarios, patrimonio \MXN{3600000}
    \end{itemize}
    
    \item \textbf{Banorte Gestión Fiduciaria}
    \begin{itemize}
        \item Contacto: [Pendiente obtener datos oficiales]
        \item Especificar: Fideicomiso inmobiliario nacional (no zona restringida)
    \end{itemize}
\end{enumerate}

\subsection{Empresas Gestoras en Mérida}

\begin{enumerate}
    \item Empresas locales especializadas en vacation rentals
    \item Operadores de Airbnb Superhosts con múltiples propiedades
    \item Property managers certificados con experiencia en zona norte
\end{enumerate}

\textbf{Criterios de evaluación}:
\begin{itemize}
    \item Track record mínimo 3 años
    \item Cartera de al menos 10 propiedades gestionadas
    \item Tasa de ocupación promedio $>$75\%
    \item Rating promedio $>$4.7 estrellas
    \item Referencias verificables
\end{itemize}

\vfill

\begin{center}
\textcolor{verdeoscuro}{\rule{0.8\textwidth}{0.5pt}}\\[0.5cm]
\textbf{\large DOCUMENTO PREPARADO PARA COPROPIETARIOS}\\
\textit{Análisis Ejecutivo - Diciembre 2025}
\end{center}

\end{document}
